\thispagestyle{empty}

\parindent=16mm

\vspace*{2cm}

\begin{flushleft}
  \begin{Large}
    \textbf{Resumen}\\
  \end{Large}
  \masterRule
\end{flushleft}

\vspace*{1cm}

El avance de la arquitectura abierta RISC-V, con su naturaleza modular y libre de licencias, se posiciona como una alternativa competitiva frente a arquitecturas propietarias como ARM. especialmente en entornos académicos y de investigación, donde la capacidad de personalización es clave. En este contexto, el uso de simulaciones \textit{hardware} para validar arquitecturas, como las realizadas con el proyecto X-HEEP, permite reducir los tiempos de desarrollo y presentación al mercado al agilizar los ciclos de desarrollo. Para poder tener un control general de la plataforma en tiempo real, el uso de los RTOS es crucial actualmente, permitiendo una gestión global sencilla. En este trabajo se abordará el desarollo de un diseño \textit{hardware} acelerador de una tarea de cómputo, validado mediante simulaciones \textit{hardware} y gestionado en todo momento mediante un RTOS.